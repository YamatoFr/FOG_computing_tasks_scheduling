\chapter*{Related Work}
\label{chap:relatedwork}

There are many existing studies and researches on the topics task scheduling, energy efficiency and AI in the context
of fog computing. The concept of fog computing was introduced by Cisco in 2012 \cite{bonomi_et_al_2012}, and since
then, many researches have been conducted to improve the performance of fog computing systems and reduce their
environmental impact.

\section*{Fog computing}

\cite{rana_abubacker_2023} and \cite{abubacker_et_al_2023} respectively investigate the concepts surrounding fog
and its applications in the context of IoT.

\section*{Task Scheduling in Fog Computing}

The task scheduling problem in fog computing has been widely studied in the literature. Many researchers have proposed
different algorithms and techniques to optimize the task scheduling process in fog computing systems. In
\cite{rjoub_bentahar_2017}, the authors propose a task scheduling algorithm based on Swarm Intelligence and Machine
Learning. This hybrid approach minimizes the execution time and the makespan and improves the performance of the load
balancing scheduling. In \cite{canali_lancellotti_2019}, an optimization model is proposed for the problem of mapping
data stream over fog nodes while considering the load of those nodes and the latency between the sensors and the nodes.
A heuristic based on genetic algorithms is then presented to address the complexity of the problem. In
\cite{yousif_et_al_2024}, the authors propose a task clustering and scheduling mechanism based on Differential
Evolution to find the optimal execution time for the tasks. The mechanism is compared to the Firefly Algorithm and
Particle Swarm Optimization, and the results show that the proposed mechanism outperforms the other two algorithms in
terms of execution time, system efficiency and stability.
% \cite{keivani_tapamo_2019,misirli_casalicchio_2024} % review/analysis

\section*{Energy Efficiency in Fog Computing}

The energy efficiency of fog computing systems has been a major concern for researchers. In \cite{he_et_al_2020}, the
authors propose two Integer Linear Programming models, where the second one aims at minimizing the energy consumption
while maximizing successfully provisioned tasks. The authors of \cite{malik_et_al_2022} provides an overview of the
energy efficiency challenges in fog computing and presents a comprehensive survey of the existing energy-efficient
techniques and algorithms. An energy-aware Metaheuristic algorithm based on the Harris Hawks Optimization algorithm,
itself based on a local search strategy for task scheduling in fog computing, is proposed in
\cite{abdel-basset_et_al_2021}.

\section*{Metaheuristics in Fog Computing}



\section*{AI in Fog Computing}

The use of AI in fog computing has been a growing trend in the literature. An efficient binary CNN with numerous skip
connections is proposed by the authors of \cite{wu_et_al_2021}. In \cite{yang_et_al_2022}, the authors designed an
intelligent energy-saving model based on CNN and a task scheduling model is designed based on the policy gradient
algorithm.