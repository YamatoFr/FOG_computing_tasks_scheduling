\chapter*{Related Work}
\label{chap:relatedwork}

There are many existing studies and researches on the topics task scheduling, energy efficiency and AI in the context
of fog computing. The concept of fog computing was introduced by Cisco in 2012 \cite{bonomi-et-al-2012}, and since
then, many researches have been conducted to improve the performance of fog computing systems and reduce their
environmental impact.

\section*{Fog computing}

Fog computing is a paradigm that extends cloud computing and services to the edge of the network.
\cite{rana-abubacker-2023} and \cite{abubacker-et-al-2023} respectively investigate the concepts surrounding fog
and its applications in the context of IoT. \cite{al-musawi-et-al-2023} offers an extensive review of the existing
work on fog computing.

\section*{Task Scheduling in Fog Computing}

The task scheduling problem in fog computing has been widely studied in the literature. Many researchers have proposed
different algorithms and techniques to optimize the task scheduling process in fog computing systems.
\cite{misirli-casalicchio-2024} offers an analysis of the existing algorithms while identifying the challenges and
research gaps. In \cite{rjoub-bentahar-2017}, the authors propose a task scheduling algorithm based on Swarm
Intelligence and Machine Learning. This hybrid approach minimizes the execution time and the makespan and improves the
performance of the load balancing scheduling. In \cite{canali-lancellotti-2019}, an optimization model is proposed for
the problem of mapping data stream over fog nodes while considering the load of those nodes and the latency between the
sensors and the nodes. A heuristic based on genetic algorithms is then presented to address the complexity of the
problem. In \cite{yousif-et-al-2024}, the authors propose a task clustering and scheduling mechanism based on
Differential Evolution to find the optimal execution time for the tasks. The mechanism is compared to the Firefly
Algorithm and Particle Swarm Optimization, and the results show that the proposed mechanism outperforms the other two
algorithms in terms of execution time, system efficiency and stability.

\section*{Energy Efficiency in Fog Computing}

The energy efficiency of fog computing systems has been a major concern for researchers. \cite{alhumaima-2020}
introduces a mathematical framework to evaluate the trade-off of fog computing systems, especially in terms of power
consumption and energy efficiency. In \cite{he-et-al-2020}, the authors propose two Integer Linear Programming models,
where the second one aims at minimizing the energy consumption while maximizing successfully provisioned tasks. The
authors of \cite{malik-et-al-2022} provides an overview of the energy efficiency challenges in fog computing and
presents a comprehensive survey of the existing energy-efficient techniques and algorithms. An energy-aware
Metaheuristic algorithm based on the Harris Hawks Optimization algorithm, itself based on a local search strategy for
task scheduling in fog computing, is proposed in \cite{abdel-basset-et-al-2021}. \cite{wang-et-al-2023} proposes an
energy efficient algorithm through an integrated computation model.

\section*{Metaheuristics in Fog Computing}

Different metaheuristic algorithms have been proposed to solve the task scheduling problem in fog computing. In
\cite{ali-et-al-2021}, the authors propose an approach based on Fuzzy Logic for real-time task scheduling in IoT
applications.

\section*{AI in Fog Computing}

The use of AI in fog computing has been a growing trend in the literature. An efficient binary CNN with numerous skip
connections is proposed by the authors of \cite{wu-et-al-2021}. In \cite{yang-et-al-2022}, the authors designed an
intelligent energy-saving model based on CNN and a task scheduling model is designed based on the policy gradient
algorithm. \cite{jing-xue-2023} present an improved convolutional neural network so solve the value function of a
Continuous Markov Decision Process model, so it can be applied to a multi-user system.
