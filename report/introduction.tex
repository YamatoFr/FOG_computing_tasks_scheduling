\chapter{Introduction}

As the number of devices grows exponentially, the demand for computing resources increases. The traditional cloud
computing model is not sufficient to handle the massive amount of data generated by these devices. Fog computing is an
alternative model that extends the computing resources to the edge of the network, closer to the devices. This model
reduces the latency and bandwidth usage, and improves the overall performance of the system. However, the increasing
demand for computing resources also raises concerns about the energy consumption and the environmental impact of fog
computing systems.

In this context, task scheduling plays a crucial role in optimizing the performance of fog computing systems. Task
scheduling algorithms aim to allocate the available resources efficiently to the tasks, while still meeting the Quality
of Service (QoS) requirements of the users. Several task scheduling algorithms have been proposed in the literature,
ranging from simple heuristics to complex optimization techniques. However, most of these algorithms focus on
minimizing the energy consumption or maximizing the performance of the system, without considering the environmental
impact of the energy sources used to power the system.

In this memoir, we aim to propose a novel task scheduling algorithm that integrate the use of green energy sources
while satisfying the QoS requirements of the users. The proposed algorithm will consider the availability of green
energy sources, mainly solar panels, and will schedule the tasks to maximize the use of green energy while minimizing
the energy consumption from the grid.

This document will start by presenting the state of the art in the field of fog computing along with the scheduling
approaches and optimization methods used in this context. Then, in Chapter~\ref{chap:problem-statement}, we will
define the problem and present the objectives of this work. Chapter~\ref{chap:iot} and \ref{chap:fog} will provide an
overview of the Internet of Things (IoT) and fog computing, respectively. In Chapter~\ref{chap:fuzzy}, we present an
offloading experiment using a fuzzy logic controller, we follow with chapter~\ref{chap:decision-trees} where we
improve the offloading experiment using decision trees. In chapter~\ref{chap:future-work}, we will discuss potential
methods to improve our work and reach the established objectives. Finally, we will conclude this document in
Chapter~\ref{chap:conclusion}.