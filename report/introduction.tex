\chapter{Introduction}

As the number of devices grows exponentially, the demand for computing resources increases. The traditional cloud
computing model is not sufficient to handle the massive amount of data generated by these devices. Fog computing is an
alternative model that extends the computing resources to the edge of the network, closer to the devices. This model
reduces the latency and bandwidth usage, and improves the overall performance of the system. However, the increasing
demand for computing resources also raises concerns about the energy consumption and the environmental impact of fog
computing systems.

In this context, task scheduling plays a crucial role in optimizing the performance of fog computing systems. Task
scheduling algorithms aim to allocate the available resources efficiently to the tasks, while still meeting the Quality
of Service (QoS) requirements of the users. Several task scheduling algorithms have been proposed in the literature,
ranging from simple heuristics to complex optimization techniques. However, most of these algorithms focus on
minimizing the energy consumption or maximizing the performance of the system, without considering the environmental
impact of the energy sources used to power the system.

In this paper, we will propose an AI-based task scheduling algorithm that integrate the use of green energy sources
while satisfying the QoS requirements of the users. The proposed algorithm will consider the availability of green
energy sources, mainly solar panels, and will schedule the tasks to maximize the use of green energy while minimizing
the energy consumption from the grid.

The rest of this paper is organized as follows. In Chapter 2, we present an overview of the existing research on task
scheduling, energy efficiency and AI in the context of fog computing. In Chapter 3, we describe the proposed task
scheduling algorithm and the integration of green energy sources. In Chapter 4, we present the experimental results
and the performance evaluation of the algorithm. Finally, in Chapter 5, we conclude the paper and discuss the future
work.