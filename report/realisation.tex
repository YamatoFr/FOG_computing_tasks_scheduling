\chapter*{Offloading using Fuzzy logic}
\label{chap:fuzzy}

\section*{Introduction}

Introduced in 1965 by Lotfi Zadeh\cite{zadeh-1965}, fuzzy logic is based on a "degree of truth" instead of a finite value, usually 0 or 1, it aims to
represent the vagueness of human language and thought. Fuzzy systems are the means to implement fuzzy logic, they are two types of fuzzy systems:
Mamdani and Sugeno, both are similar but differ in the way the output is determined. The most common one Mamdani and it's the one we will be using in
this project. The Mamdani fuzzy system follows three steps:
\begin{itemize}
	\item the inputs are fuzzified into fuzzy membership functions,
	\item a set of rules are applied to the fuzzy inputs to determine the fuzzy output,
	\item the fuzzy output is defuzzified to get a crisp value.
\end{itemize}

\subsection*{Fuzzification}

Fuzzification is the process of converting a crisp value into a fuzzy value, this is done by assigning a membership function to the input value.
The membership function is a curve that defines how much the input value belongs to a certain fuzzy set. The most common membership functions are
the triangular and trapezoidal functions, they are defined by three or four parameters respectively. The triangular function is defined by the
parameters $a$, $b$ and $c$ and is given by:
\begin{equation}
	\mu(x) = \begin{cases}
		0                   & \text{if } x \leq a,        \\
		\frac{x - a}{b - a} & \text{if } a \leq x \leq b, \\
		\frac{c - x}{c - b} & \text{if } b \leq x \leq c, \\
		0                   & \text{if } x \geq c.
	\end{cases}
\end{equation}

The trapezoidal function is defined by the parameters $a$, $b$, $c$ and $d$ and is given by:
\begin{equation}
	\mu(x) = \begin{cases}
		0                   & \text{if } x \leq a,        \\
		\frac{x - a}{b - a} & \text{if } a \leq x \leq b, \\
		1                   & \text{if } b \leq x \leq c, \\
		\frac{d - x}{d - c} & \text{if } c \leq x \leq d, \\
		0                   & \text{if } x \geq d.
	\end{cases}
\end{equation}

\subsection*{Rule evaluation}

The rule evaluation is the process of determining the fuzzy output based on the fuzzy inputs and a set of rules. The rules are defined by two parts:
the "if" or antecedent part and the "then" or consequent part. The antecedent part deals with inputs, it can either be a single input or a combination
of inputs, the combination can be done using the logical operators "and" and "or". The consequent part deals with the output. In the context of this
project, the rules are usually of the form "if Bandwidth is low then processing is local". The rules are generally defined by the user and are based
on their knowledge of the system.

\subsection*{Defuzzification}

Defuzzification is the process of converting a fuzzy output into a crisp value, there are several methods to do this, the most common one is the
centroid method. The centroid method calculates the center of mass of the fuzzy output, this is done by taking the weighted average of the output
values. The weighted average is calculated by taking the sum of the product of the output value and its membership value divided by the sum of the
membership values. The formula for the centroid method is given by:
\begin{equation}
	y = \frac{\sum_{i} \mu_i \cdot y_i}{\sum_{i} \mu_i}.
\end{equation}
Where $y$ is the crisp output, $\mu_i$ is the membership value of the output value $y_i$.