\chapter{Internet of Things (IoT)}
\label{chap:iot}

The Internet of Things (IoT) is a network of interconnected devices that can communicate with each other and exchange
data. These devices can be anything from smartphones and laptops to sensors and actuators. The IoT is a rapidly growing
field with many applications in various industries like healthcare, agriculture, transportation, and manufacturing. The
goal of the IoT is to make our lives easier by automating tasks and providing real-time information. It is made
possible by advances in wireless communication, sensor technology, and cloud computing. These technologies allow devices
to connect to the internet and share data with each other and the cloud. The IoT has many benefits like improved
efficiency, reduced costs, and increased safety. However, it also has some challenges like security, privacy, and
interoperability.

\section{History}
\label{sec:iot-history}

The concept and the term "Internet of Things" can be traced back to the 1985 during a speech by Peter T. Lewis to the
Congressional Black Caucus Foundation.\cite{chetansharma-2016} He said: "The Internet of Things, or IoT, is the
integration of people, processes and technology with connectable devices and sensors to enable remote monitoring,
status, manipulation and evaluation of trends of such devices." The term was later popularized by Kevin Ashton in
1999.\cite{ashton-2009}

The first internet-connected (previously known as ARPANET) device was a Coca-Cola vending machine at Carnegie Mellon
University in 1982,\cite{cscmuedu} it was able to report its inventory and temperature to the network. Since then,
the IoT has grown rapidly with the advent of wireless communication, sensor technology, and cloud computing.

Cisco Systems estimated that the number of internet-connected devices exceeded the number of people on Earth between
2008 and 2009,\cite{evans-2011} marking the beginning of the IoT era. Today, there are billions of IoT devices around
the world, and the number will continue to grow as more devices become connected.

\section{Architecture}
\label{sec:iot-architecture}

The IoT architecture consists of three layers: the perception layer, the network layer, and the application layer.
\begin{itemize}
	\item The perception layer is where the devices are located, it includes sensors, actuators, and other devices that
	      collect data and interact with the physical world.
	\item The network layer is where the devices communicate with each other and the cloud, it includes wireless
	      communication technologies like Wi-Fi, Bluetooth, and Zigbee.
	\item The application layer is where the data is processed and analyzed, it includes cloud computing services like
	      Amazon Web Services (AWS), Microsoft Azure, and Google Cloud Platform.
\end{itemize}

The IoT architecture is designed to be scalable, flexible, and secure. It allows devices to connect to the internet and
share data with each other and the cloud. This enables new applications and services that were not possible before. The
IoT architecture is still evolving with new technologies like edge computing and fog computing. These technologies bring
the cloud closer to the devices and reduce latency and bandwidth requirements.

\section{Applications}
\label{sec:iot-applications}

\subsection{Home automation}
\label{subsec:iot-home-automation}

One of the most popular applications of the IoT is home automation, it allows homeowners to control many aspects of their
home like lighting, heating, and security. Smart devices like thermostats, light bulbs, and cameras can be connected to
the internet and controlled remotely using a smartphone or voice assistant.

Home automation can help homeowners save energy, improve security, and make their lives more convenient. It is also a
growing market with many companies offering smart home products and services, like Google Home, Amazon Echo, and Apple
HomeKit. Non-proprietary solutions are also available, like Home Assistant, OpenHAB, and Domoticz.

\subsection{Healthcare}
\label{subsec:iot-healthcare}

IoT has many applications in healthcare, it can be used to monitor patients, track medications, and manage chronic diseases.
Wearable devices like smartwatches and fitness trackers can collect data on a patient's heart rate, blood pressure, and
activity level. This data can be sent to a healthcare provider for analysis and diagnosis. The IoT can also be used to monitor
patients in hospitals and nursing homes, it can alert healthcare providers if a patient's condition changes. The IoT can help
improve patient outcomes, reduce costs, and increase access to care.

However, it also raises concerns about privacy, security, and data ownership. The Health Insurance Portability and Accountability
Act (HIPAA)[\href{https://www.hhs.gov/hipaa/for-professionals/privacy/index.html}{HIPAA privacy rule}] in the United States
and the General Data Protection Regulation
(GDPR)[\href{https://www.cnil.fr/fr/reglement-europeen-protection-donnees/chapitre1#Article4}{article 4-15}] in the European
Union are two regulations that govern the use of healthcare data.

\subsection{Energy management}
\label{subsec:iot-energy-management}

Homeowners can monitor and control their energy usage using with many IoT devices, this principle can be extended to a larger
scale with smart grids. Smart grids are electrical grids that use IoT devices to monitor and control the flow of electricity.
They can help utilities reduce costs, improve reliability, and increase efficiency. They can also help reduce greenhouse gas
emissions and promote renewable energy sources. It is a growing field with many companies offering smart grid products and
services, like Siemens, ABB, and Schneider Electric.

\chapter{FOG computing}
\label{chap:fog}

FOG computing was introduced by Cisco in 2012 as a way to extend cloud computing to the edge of the network. The goal
of FOG computing is to bring the cloud closer to the end-users, this is done by placing the cloud's resources at the
edge of the network thanks to a wider geographical distribution.

\section{Concept}
\label{sec:fog-concept}

FOG computing is a virtualized and decentralized computing platform that acts as an intermediary between the cloud and
the edge devices. The word "fog" is used to describe the cloud-like properties being closer to the "ground", the ground
being the edge devices. Many of these devices will generate a huge amount or raw data that needs to be processed in real
time, this is where FOG computing comes in. Instead of sending all the data to the cloud for processing, the data is
processed on FOG nodes. The nodes can be anything from resource-rich servers to resource-constrained devices like
gateways, routers or other end devices.

It is designed to reduce latency, provide computing resources, storage and network services to the edge devices. It also
provides location awareness, mobility support, and real-time data processing. FOG computing is ideal for applications that
require real-time processing, like autonomous vehicles, industrial automation, and augmented reality. FOG has also some
drawbacks, it has a higher power consumption than edge computing, security and privacy concerns are also a major challenge.
It also requires a higher bandwidth and more complex network infrastructure.

However, FOG computing is not to be confused with edge computing, while they share some similarities, they are not the
same. Edge computing is a subset of FOG computing, the processing is done at the edge of the network, while FOG computing
allows the processing to be done anywhere between the edge and the cloud.\cite{onlogic,gfg,baeldung}

\section{Architecture}
\label{sec:fog-architecture}

The architecture of a FOG network is shown in figure \ref{fig:fog-architecture}. It consists of three layers:
\begin{itemize}
	\item The edge layer is where the end devices are located, it includes sensors, actuators, and other devices that
	      generate data.
	\item The FOG layer is where the FOG nodes are located, it includes servers, gateways, and other devices that
	      process data.
	\item The cloud layer is where the cloud data centers are located, it includes servers, storage, and network
	      services that store and process data.
\end{itemize}

\begin{figure}[H]
	\centering
	\includegraphics[width=0.90\textwidth]{../images/FOG_network_ink.png}
	\caption{FOG computing architecture.}
	\label{fig:fog-architecture}
\end{figure}

\chapter{Offloading using Fuzzy logic}
\label{chap:fuzzy}

\section{Explaining fuzzy logic}
\label{sec:fuzzy-explanation}

Introduced in 1965 by Lotfi Zadeh,\cite{zadeh-1965} fuzzy logic is based on a "degree of truth" instead of a finite
value, usually 0 or 1, it aims to represent the vagueness of human language and thought. Fuzzy systems are the means
to implement fuzzy logic, they are two types of fuzzy systems: Mamdani and Sugeno, both are similar but differ in the
way the output is determined. The most common one Mamdani and it's the one we will be using in this project. The
Mamdani fuzzy system follows three steps:

\begin{itemize}
	\item The inputs are fuzzified into fuzzy membership functions,
	\item a set of rules are applied to the fuzzy inputs to determine the fuzzy output,
	\item the fuzzy output is defuzzified to get a crisp value.
\end{itemize}

While Sugeno includes the defuzzification step in the rule evaluation step, this system works well with optimization
algorithms and is more computationally efficient than the Mamdani system. But for this project we will be using the
Mamdani system.

\subsection{Fuzzification}
\label{subsec:fuzzy-fuzzification}

Fuzzification is the process of converting a crisp value into a fuzzy value, this is done by assigning a membership
function to the input value. The membership function is a curve that defines how much the input value belongs to a
certain fuzzy set. The most common membership functions are the triangular and trapezoidal functions, they are defined
by three or four parameters respectively. The triangular function is defined by the parameters $a$, $b$ and $c$ and
is given by:

\begin{equation}
	\mu(x) = \begin{cases}
		0                   & \text{if } x \leq a,        \\
		\frac{x - a}{b - a} & \text{if } a \leq x \leq b, \\
		\frac{c - x}{c - b} & \text{if } b \leq x \leq c, \\
		0                   & \text{if } x \geq c.
	\end{cases}
\end{equation}

The trapezoidal function is defined by the parameters $a$, $b$, $c$ and $d$ and is given by:

\begin{equation}
	\mu(x) = \begin{cases}
		0                   & \text{if } x \leq a,        \\
		\frac{x - a}{b - a} & \text{if } a \leq x \leq b, \\
		1                   & \text{if } b \leq x \leq c, \\
		\frac{d - x}{d - c} & \text{if } c \leq x \leq d, \\
		0                   & \text{if } x \geq d.
	\end{cases}
\end{equation}

\subsection{Rule evaluation}
\label{subsec:fuzzy-rule-evaluation}

The rule evaluation is the process of determining the fuzzy output based on the fuzzy inputs and a set of rules. The
rules are defined by two parts: the "if" or antecedent part and the "then" or consequent part. The antecedent part
deals with inputs, it can either be a single input or a combination of inputs, the combination can be done using the
logical operators "and" and "or". The consequent part deals with the output. In the context of this project, the rules
are usually of the form "if Bandwidth is low then processing is local". The rules are generally defined by the user
and are based on their knowledge of the system.

\subsection{Aggregation}
\label{subsec:fuzzy-aggregation}

The aggregation is the process of combining the fuzzy outputs from the rules to get a single fuzzy output. This process
relies on T-conorms, and must satisfy the following properties:

\begin{itemize}
	\item Commutativity: $x * y = y * x$,
	\item Associativity: $x * (y * z) = (x * y) * z$,
	\item Monotony: $x \leq y \implies x * z \leq y * z$,
	\item Neutrality of 0: $x * 0 = x$ for $x \in [0, 1]$.
\end{itemize}

They are also \textit{positive reinforcement} operators:

\begin{equation}
	f(x_1, \cdots, x_n) \leq max[x_i] \forall x_i \geq 0.5
\end{equation}

As opposed to T-norms which are \textit{negative reinforcement} operators:

\begin{equation}
	f(x_1, \cdots, x_n) \geq min[x_i] \forall x_i \leq 0.5
\end{equation}

The most common T-conorms are the maximum, the probabilistic sum and the bounded sum. They are defined as follows:

\begin{minipage}{0.9624\textwidth}
	\begin{equation}
		\text{Maximum: } x \oplus y = \max(x, y)
	\end{equation}
	\begin{equation}
		\text{Probabilistic sum: } x \oplus y = x + y - x \cdot y
	\end{equation}
	\begin{equation}
		\text{Bounded sum: } x \oplus y = \min(x + y, 1)
	\end{equation}
\end{minipage}

\subsection{Defuzzification}
\label{subsec:fuzzy-defuzzification}

Defuzzification is the process of converting a fuzzy output into a crisp value, there are several methods to do this,
the most common one is the centroid method. The centroid method calculates the center of mass of the fuzzy output, this
is done by taking the weighted average of the output values. The weighted average is calculated by taking the sum of
the product of the output value and its membership value divided by the sum of the membership values. The formula for
the centroid method is given by:

\begin{equation}
	y = \frac{\sum_{i} \mu_i \cdot y_i}{\sum_{i} \mu_i}
\end{equation}

Where $y$ is the crisp output, $\mu_i$ is the membership value of the output value $y_i$. The centroid method is the
most common method because it is simple and easy to implement. However, it is not always the best method, other methods
like the mean of maximum and the largest of maximum can be used depending on the application.

\section{Fuzzy system for offloading}
\label{sec:fuzzy-offloading}

\subsection{Setting up the engine}
\label{subsec:fuzzy-setup}

To demonstrate the use of fuzzy logic in offloading, we wrote two simple programs in Python. The goal was to recreate
the experiment done by Hari et al.\cite{Hari-et-al-2018} The first program uses the pyfuzzylite\cite{fuzzylite} library
to implement a fuzzy engine with all the variables and rules needed to determine the offloading decision. The second
program uses the NumPy library to generate random values for the inputs and then uses the fuzzy engine to determine
the offloading decision. The fuzzy engine is defined by the variables shown in table \ref{tab:fuzzy-input} and
\ref{tab:fuzzy-output}.

\begin{table}[H]
	\centering
	\begin{tabular}{|c|c|c|c|c|}
		\hline
		Name                & Range    & Fuzzy set & Membership function & Parameters         \\
		\hline
		Bandwidth           & [0, 100] & bw_low    & trapezoidal         & 0, 20, 30, 40      \\
		(in Mbps)           &          & bw_med    & trapezoidal         & 35, 45, 60, 70     \\
		                    &          & bw_high   & trapezoidal         & 65, 75, 90, 100    \\
		\hline
		Data size           & [0, 600] & data_low  & trapezoidal         & 0, 0, 230, 360     \\
		(in KB)             &          & data_med  & trapezoidal         & 250, 350, 470, 590 \\
		                    &          & data_high & trapezoidal         & 450, 540, 600, 600 \\
		\hline
		Residual battery    & [0, 100] & bat_low   & trapezoidal         & 0, 0, 25, 35       \\
		charge (in \%)      &          & bat_med   & trapezoidal         & 25, 40, 60, 75     \\
		                    &          & bat_high  & trapezoidal         & 60, 75, 100, 100   \\
		\hline
		Load                & [0, 100] & load_low  & trapezoidal         & 0, 0, 25, 40       \\
		(in \%)             &          & load_med  & trapezoidal         & 35, 45, 60, 70     \\
		                    &          & load_high & trapezoidal         & 65, 80, 100, 100   \\
		\hline
		Memory              & [0, 100] & mem_low   & trapezoidal         & 0, 0, 25, 40       \\
		(in \%)             &          & mem_med   & trapezoidal         & 35, 45, 60, 70     \\
		                    &          & mem_high  & trapezoidal         & 65, 80, 100, 100   \\
		\hline
		Virtual machines    & [0, 50]  & vm_low    & trapezoidal         & 0, 0, 15, 20       \\
		available           &          & vm_med    & trapezoidal         & 15, 22, 37, 40     \\
		                    &          & vm_high   & trapezoidal         & 30, 35, 50, 50     \\
		\hline
		Number              & [0, 100] & user_low  & trapezoidal         & 0, 0, 25, 40       \\
		of concurrent users &          & user_med  & trapezoidal         & 30, 40, 60, 70     \\
		                    &          & user_high & trapezoidal         & 60, 75, 100, 100   \\
		\hline
	\end{tabular}
	\caption{Input variables for the fuzzy engine.}
	\label{tab:fuzzy-input}
\end{table}

\begin{table}[H]
	\centering
	\begin{tabular}{|c|c|c|c|c|}
		\hline
		Name                & Range    & Fuzzy set & Membership function & Parameters      \\
		\hline
		Offloading decision & [0, 100] & local     & trapezoidal         & 0, 12, 24, 48   \\
		                    &          & remote    & trapezoidal         & 36, 60, 72, 100 \\
		\hline
	\end{tabular}
	\caption{Output variable for the fuzzy engine.}
	\label{tab:fuzzy-output}
\end{table}

The original paper by Hari et al. did not provide all the rules used, so we had to come up with our own rules. We
established them based on our understanding of the system and the variables. Unlike the original paper, where the
rules only combined the bandwidth with one other variable, we decided to combine the bandwidth with all the relevant
variables. The rules are shown in table \ref{tab:fuzzy-rules}.

\begin{table}[H]
	\centering
	\resizebox{\textwidth}{!}{%
		\begin{tabular}{|l|l|}
			\hline
			   & Rules                                                                                                  \\
			\hline
			R1 & IF Bandwidth is bw_low                                                                                 \\
			   & THEN Processing local_processing                                                                       \\
			\hline
			R2 & IF Bandwidth is not bw_low and Datasize is not data_low                                                \\
			   & THEN Processing is remote_processing                                                                   \\
			\hline
			R3 & IF Bandwidth is not bw_low and Datasize is data_low and (Load is load_high or Memory is mem_low)       \\
			   & THEN Processing is remote_processing                                                                   \\
			\hline
			R4 & IF Bandwidth is not bw_low and Datasize is data_low and (NB_concurrent_users is user_low or Memory is
			mem_low)                                                                                                    \\
			   & THEN Processing is remote_processing                                                                   \\
			\hline
			R5 & IF Bandwidth is not bw_low and Datasize is data_low and NB_concurrent_users is not user_low and Memory
			is not mem_low and Load is load_low                                                                         \\
			   & THEN Processing is local_processing                                                                    \\
			\hline
			R6 & IF Bandwidth is not bw_low and Datasize is data_low and NB_concurrent_users is not user_low and Memory
			is not mem_low and Load is not load_low                                                                     \\
			   & THEN Processing is remote_processing                                                                   \\
			\hline
		\end{tabular}}
	\caption{Rules for the fuzzy engine.}
	\label{tab:fuzzy-rules}
\end{table}

From these rules we can see that two variables have no impact on the offloading decision, the residual battery charge
and the number of virtual machines available. So we decided to remove them from the fuzzy engine.

\subsection{Mean 3\texorpdfstring{$\Pi$}{Pi} aggregation operator (M3\texorpdfstring{$\Pi$}{Pi})}

The standard 3$\Pi$ operator was introduced by Yager and Rybalov \cite{yager-rybalov-1998} in 1988, it is a
generalization of the probabilistic sum. The 3$\Pi$ operator is defined by the following formula:

\begin{equation}
	\Pi(x_1, \cdots, x_n) = \frac{\Pi_{j=1}^n x_j}{\Pi_{j=1}^n x_j + \Pi_{j=1}^n (1 - x_j)}
\end{equation}

Unlike T-norms and T-conorms, the 3$\Pi$ operator is \textit{full reinforced} which means that it satisfies the
properties of both positive and negative reinforcement operators.

We decided to implement a M3$\Pi$ aggregation operator which is derived from the 3$\Pi$ operator, it is defined by the
following formula:

\begin{equation}
	M3\Pi(x_1, \cdots, x_n) = \frac{\Pi_{j=1}^n (x_j)^{(1/n)}}{\Pi_{j=1}^n (x_j)^{(1/n)} + \Pi_{j=1}^n (1 - x_j)^{(1/n)}}
\end{equation}

The goal was to improve the decision-making process. However, we were unable to implement the M3$\Pi$ operator as it
would require to modify a large part of the \textit{pyfuzzylite} library. We decided to shift our focus towards other
parts of the project.

\subsection{Partial results}
\label{subsec:fuzzy-results}

Despite the absence of the M3$\Pi$ operator, we still ran the fuzzy engine with three different aggregation operators:
maximum, probabilistic sum and bounded sum. Samples of the results are shown in tables \ref{tab:fuzzy-results-max},
\ref{tab:fuzzy-results-prob} and \ref{tab:fuzzy-results-bounded}.

\begin{table}[H]
	\centering
	\resizebox{\textwidth}{!}{%
		\begin{tabular}{|l|l|l|l|l|l|l|l|l|}
			\hline
			Bandwidth & Data size & Number of        & Memory & Load & Processing & Fuzzy output                                     \\
			          &           & concurrent users &        &      &            &                                                  \\
			\hline
			30        & 180       & 42               & 38     & 66   & 21.575     & 1.000/local_processing + 0.000/remote_processing \\
			77        & 462       & 44               & 96     & 8    & 67.539     & 0.000/local_processing + 1.000/remote_processing \\
			3         & 21        & 45               & 75     & 50   & 60.732     & 0.150/local_processing + 0.850/remote_processing \\
			38        & 229       & 51               & 31     & 66   & 57.565     & 0.200/local_processing + 0.600/remote_processing \\
			50        & 302       & 17               & 96     & 92   & 68.116     & 0.000/local_processing + 0.554/remote_processing \\
			\hline
		\end{tabular}}
	\caption{Sample results from the fuzzy engine with the Maximum aggregation operator.}
	\label{tab:fuzzy-results-max}

	\bigskip

	\centering
	\resizebox{\textwidth}{!}{%
		\begin{tabular}{|l|l|l|l|l|l|l|l|l|}
			\hline
			Bandwidth & Data size & Number of        & Memory & Load & Processing & Fuzzy output                                     \\
			          &           & concurrent users &        &      &            &                                                  \\
			\hline
			30        & 180       & 42               & 38     & 66   & 21.575     & 1.000/local_processing + 0.000/remote_processing \\
			77        & 462       & 44               & 96     & 8    & 67.539     & 0.000/local_processing + 1.000/remote_processing \\
			3         & 21        & 45               & 75     & 50   & 60.443     & 0.150/local_processing + 0.850/remote_processing \\
			38        & 229       & 51               & 31     & 66   & 61.27      & 0.200/local_processing + 0.904/remote_processing \\
			50        & 302       & 17               & 96     & 92   & 68.314     & 0.000/local_processing + 0.863/remote_processing \\
			\hline
		\end{tabular}}
	\caption{Sample results from the fuzzy engine with the Probabilistic sum
		aggregation operator (note: the library calls it Algebraic sum).}
	\label{tab:fuzzy-results-prob}

	\bigskip

	\centering
	\resizebox{\textwidth}{!}{%
		\begin{tabular}{|l|l|l|l|l|l|l|l|l|}
			\hline
			Bandwidth & Data size & Number of        & Memory & Load & Processing & Fuzzy output                                     \\
			          &           & concurrent users &        &      &            &                                                  \\
			\hline
			30        & 180       & 42               & 38     & 66   & 21.575     & 1.000/local_processing + 0.000/remote_processing \\
			77        & 462       & 44               & 96     & 8    & 67.539     & 0.000/local_processing + 1.000/remote_processing \\
			3         & 21        & 45               & 75     & 50   & 60.275     & 0.150/local_processing + 0.850/remote_processing \\
			38        & 229       & 51               & 31     & 66   & 61.855     & 0.200/local_processing + 1.000/remote_processing \\
			50        & 302       & 17               & 96     & 92   & 67.903     & 0.000/local_processing + 1.000/remote_processing \\
			\hline
		\end{tabular}}
	\caption{Sample results from the fuzzy engine with the Bounded sum aggregation operator.}
	\label{tab:fuzzy-results-bounded}
\end{table}

\chapter{Decision trees}
\label{chap:decision-trees}

In an attempt to improve the offloading decision-making process, we decided to use the Weka\cite{weka} software to create
a decision tree. We will use the J48 algorithm which is a Java implementation of the C4.5 algorithm. The goal is to was to
improve upon the decision-making process of the fuzzy engine.

\section{The Weka software}
\label{sec:weka}

Weka is a collection of machine learning algorithms for data processing and predictive modeling. It is written in Java and
is distributed under the GNU General Public License. The software is designed to be user-friendly and easy to use, thanks
to its graphical user interface. It is widely used in academia and industry for data mining, machine learning, and predictive
analytics. Weka provides a wide range of algorithms for classification, regression, clustering, etc. The software also
provides access to SQL databases with Java Database Connectivity (JDBC) and can also accept multiple data formats like
ARFF, CSV, C4.5, etc.

\subsection{The J48 algorithm}
\label{subsec:j48}

The J48 algorithm is a decision tree algorithm, it allows the creation of a decision tree from a set of training data. The
algorithm works by recursively splitting the data into subsets based on the values of the input variables. The goal is to
create a tree that can predict the value of the output variable based on the input variables. The algorithm uses the
information gain to determine the best split at each node, the information gain is a measure of the reduction in entropy.

The training data is a set of already classified data, it contains the values of the input variables and the corresponding
value of the output variable and the class it belongs to. The algorithm uses this data to build the decision tree, at each
node, it selects the attribute that split the data into the best subsets. The process is repeated until all the data is
classified. The decision tree can then be used to predict the value of the output variable for new data.

Decision trees are used to represent the decision-making process in a graphical form, they are easy to understand and
interpret. They can also be used to identify the most important variables in the data and to detect patterns and
relationships between the variables.

\subsubsection{The decision tree}
\label{subsubsec:decision-tree}

The resulting decision tree is composed of nodes, leaves and branches:
\begin{itemize}
	\item The nodes represent the attributes where the data was split,
	\item the branches represent the value of the attribute,
	\item the leaves represent the class that the data belongs to.
\end{itemize}

When building the tree, the missing values are ignored but can be predicted by using the attribute value that is most
common in the training data. The tree can be pruned to improve its accuracy and reduce its complexity. Pruning is the
process of removing nodes that do not improve the accuracy of the tree. The decision tree can be visualized using the
graphical user interface of Weka. The tree can also be exported as a text file for further analysis.

