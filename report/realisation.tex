\chapter{Offloading using Fuzzy logic}
\label{chap:fuzzy}

\section{Introduction}

Introduced in 1965 by Lotfi Zadeh\cite{zadeh-1965}, fuzzy logic is based on a "degree of truth" instead of a finite value, usually 0 or 1, it aims to
represent the vagueness of human language and thought. Fuzzy systems are the means to implement fuzzy logic, they are two types of fuzzy systems:
Mamdani and Sugeno, both are similar but differ in the way the output is determined. The most common one Mamdani and it's the one we will be using in
this project. The Mamdani fuzzy system follows three steps:
\begin{itemize}
	\item the inputs are fuzzified into fuzzy membership functions,
	\item a set of rules are applied to the fuzzy inputs to determine the fuzzy output,
	\item the fuzzy output is defuzzified to get a crisp value.
\end{itemize}

\subsection{Fuzzification}

Fuzzification is the process of converting a crisp value into a fuzzy value, this is done by assigning a membership function to the input value.
The membership function is a curve that defines how much the input value belongs to a certain fuzzy set. The most common membership functions are
the triangular and trapezoidal functions, they are defined by three or four parameters respectively. The triangular function is defined by the
parameters $a$, $b$ and $c$ and is given by:
\begin{equation}
	\mu(x) = \begin{cases}
		0                   & \text{if } x \leq a,        \\
		\frac{x - a}{b - a} & \text{if } a \leq x \leq b, \\
		\frac{c - x}{c - b} & \text{if } b \leq x \leq c, \\
		0                   & \text{if } x \geq c.
	\end{cases}
\end{equation}

The trapezoidal function is defined by the parameters $a$, $b$, $c$ and $d$ and is given by:
\begin{equation}
	\mu(x) = \begin{cases}
		0                   & \text{if } x \leq a,        \\
		\frac{x - a}{b - a} & \text{if } a \leq x \leq b, \\
		1                   & \text{if } b \leq x \leq c, \\
		\frac{d - x}{d - c} & \text{if } c \leq x \leq d, \\
		0                   & \text{if } x \geq d.
	\end{cases}
\end{equation}

\subsection{Rule evaluation}

The rule evaluation is the process of determining the fuzzy output based on the fuzzy inputs and a set of rules. The rules are defined by two parts:
the "if" or antecedent part and the "then" or consequent part. The antecedent part deals with inputs, it can either be a single input or a combination
of inputs, the combination can be done using the logical operators "and" and "or". The consequent part deals with the output. In the context of this
project, the rules are usually of the form "if Bandwidth is low then processing is local". The rules are generally defined by the user and are based
on their knowledge of the system.

\subsection{Aggregation}

The aggregation is the process of combining the fuzzy outputs from the rules to get a single fuzzy output. This process relies on S-Norms, and
must satisfy the following properties:
\begin{itemize}
	\item Commutativity: $x * y = y * x$,
	\item Associativity: $x * (y * z) = (x * y) * z$,
	\item Monotony: $x \leq y \implies x * z \leq y * z$,
	\item Neutrality of 0: $x * 0 = x$ for $x \in [0, 1]$.
\end{itemize}

The most common S-Norms are the maximum and the probabilistic sum. The maximum and probabilistic sum are defined by the following formulas:

\begin{minipage}{0.9685\textwidth}
	\begin{equation}
		\text{Maximum: } x * y = \max(x, y)
	\end{equation}
	\begin{equation}
		\text{Probabilistic sum: } x * y = x + y - x \cdot y
	\end{equation}
\end{minipage}

\subsection{Defuzzification}

Defuzzification is the process of converting a fuzzy output into a crisp value, there are several methods to do this, the most common one is the
centroid method. The centroid method calculates the center of mass of the fuzzy output, this is done by taking the weighted average of the output
values. The weighted average is calculated by taking the sum of the product of the output value and its membership value divided by the sum of the
membership values. The formula for the centroid method is given by:
\begin{equation}
	y = \frac{\sum_{i} \mu_i \cdot y_i}{\sum_{i} \mu_i}
\end{equation}
Where $y$ is the crisp output, $\mu_i$ is the membership value of the output value $y_i$. The centroid method is the most common method because it
is simple and easy to implement. However, it is not always the best method, other methods like the mean of maximum and the largest of maximum can
be used depending on the application.

\section{Fuzzy system for offloading}

\subsection{Setting up the engine}

To demonstrate the use of fuzzy logic in offloading, we wrote two simple programs in Python. The goal was to recreate the experiment done by Hari et
al.\cite{Hari-et-al-2018}. The first program uses the pyfuzzylite\cite{fuzzylite} library to implement a fuzzy engine with all the variables and
rules needed to determine the offloading decision. The second program uses the NumPy library to generate random values for the inputs and then uses
the fuzzy engine to determine the offloading decision. The fuzzy engine is defined by the variables shown in table \ref{tab:fuzzy-input} and
\ref{tab:fuzzy-output}.

\begin{table}
	\centering
	\begin{tabular}{|c|c|c|c|c|}
		\hline
		Name                & Range    & Fuzzy set & Membership function & Parameters         \\
		\hline
		Bandwidth           & [0, 100] & bw_low    & trapezoidal         & 0, 20, 30, 40      \\
		(in Mbps)           &          & bw_med    & trapezoidal         & 35, 45, 60, 70     \\
		                    &          & bw_high   & trapezoidal         & 65, 75, 90, 100    \\
		\hline
		Data size           & [0, 600] & data_low  & trapezoidal         & 0, 0, 230, 360     \\
		(in KB)             &          & data_med  & trapezoidal         & 250, 350, 470, 590 \\
		                    &          & data_high & trapezoidal         & 450, 540, 600, 600 \\
		\hline
		Residual battery    & [0, 100] & bat_low   & trapezoidal         & 0, 0, 25, 35       \\
		charge (in \%)      &          & bat_med   & trapezoidal         & 25, 40, 60, 75     \\
		                    &          & bat_high  & trapezoidal         & 60, 75, 100, 100   \\
		\hline
		Load                & [0, 100] & load_low  & trapezoidal         & 0, 0, 25, 40       \\
		(in \%)             &          & load_med  & trapezoidal         & 35, 45, 60, 70     \\
		                    &          & load_high & trapezoidal         & 65, 80, 100, 100   \\
		\hline
		Memory              & [0, 100] & mem_low   & trapezoidal         & 0, 0, 25, 40       \\
		(in \%)             &          & mem_med   & trapezoidal         & 35, 45, 60, 70     \\
		                    &          & mem_high  & trapezoidal         & 65, 80, 100, 100   \\
		\hline
		Virtual machines    & [0, 50]  & vm_low    & trapezoidal         & 0, 0, 15, 20       \\
		available           &          & vm_med    & trapezoidal         & 15, 22, 37, 40     \\
		                    &          & vm_high   & trapezoidal         & 30, 35, 50, 50     \\
		\hline
		Number              & [0, 100] & user_low  & trapezoidal         & 0, 0, 25, 40       \\
		of concurrent users &          & user_med  & trapezoidal         & 30, 40, 60, 70     \\
		                    &          & user_high & trapezoidal         & 60, 75, 100, 100   \\
		\hline
	\end{tabular}
	\caption{Input variables for the fuzzy engine.}
	\label{tab:fuzzy-input}
\end{table}

\begin{table}
	\centering
	\begin{tabular}{|c|c|c|c|c|}
		\hline
		Name                & Range    & Fuzzy set & Membership function & Parameters      \\
		\hline
		Offloading decision & [0, 100] & local     & trapezoidal         & 0, 12, 24, 48   \\
		                    &          & remote    & trapezoidal         & 36, 60, 72, 100 \\
		\hline
	\end{tabular}
	\caption{Output variable for the fuzzy engine.}
	\label{tab:fuzzy-output}
\end{table}

The original paper by Hari et al. did not provide all the rules used, so we had to come up with our own rules. We established them based our
understanding of the system and the variables. Unlike the original paper, where the rules only combined the bandwidth with one other variable, we
decided to combine the bandwidth with all the relevant variables. The rules are shown in table \ref{tab:fuzzy-rules}.

\begin{table}
	\centering
	\resizebox{\textwidth}{!}{%
		\begin{tabular}{|l|l|}
			\hline
			   & Rules                                                                                                                                          \\
			\hline
			R1 & IF Bandwidth is bw_low                                                                                                                         \\
			   & THEN Processing local_processing                                                                                                               \\
			\hline
			R2 & IF Bandwidth is not bw_low and Datasize is not data_low                                                                                        \\
			   & THEN Processing is remote_processing                                                                                                           \\
			\hline
			R3 & IF Bandwidth is not bw_low and Datasize is data_low and (Load is load_high or Memory is mem_low)                                               \\
			   & THEN Processing is remote_processing                                                                                                           \\
			\hline
			R4 & IF Bandwidth is not bw_low and Datasize is data_low and (NB_concurrent_users is user_low or Memory is mem_low)                                 \\
			   & THEN Processing is remote_processing                                                                                                           \\
			\hline
			R5 & IF Bandwidth is not bw_low and Datasize is data_low and NB_concurrent_users is not user_low and Memory is not mem_low and Load is load_low     \\
			   & THEN Processing is local_processing                                                                                                            \\
			\hline
			R6 & IF Bandwidth is not bw_low and Datasize is data_low and NB_concurrent_users is not user_low and Memory is not mem_low and Load is not load_low \\
			   & THEN Processing is remote_processing                                                                                                           \\
			\hline
		\end{tabular}}
	\caption{Rules for the fuzzy engine.}
	\label{tab:fuzzy-rules}
\end{table}

From these rules we can see that two variables have no impact on the offloading decision, the residual battery charge and the number of virtual
machines available.

\subsection{Mean 3\texorpdfstring{$\Pi$}{Pi} aggregation operator (M3\texorpdfstring{$\Pi$}{Pi})}

The standard 3$\Pi$ operator was introduced by Yager and Rybalov \cite{yager-rybalov-1998} in 1988, it is a generalization of the probabilistic sum.
The 3$\Pi$ operator is defined by the following formula:
\begin{equation}
	\Pi(x_1, \cdots, x_n) = \frac{\Pi_{j=1}^n x_j}{\Pi_{j=1}^n x_j + \Pi_{j=1}^n (1 - x_j)}
\end{equation}

We decided to implement a M3$\Pi$ aggregation operator which is derived from the 3$\Pi$ operator, it is defined by the following formula:
\begin{equation}
	M3\Pi(x_1, \cdots, x_n) = \frac{\Pi_{j=1}^n (x_j)^{(1/n)}}{\Pi_{j=1}^n (x_j)^{(1/n)} + \Pi_{j=1}^n (1 - x_j)^{(1/n)}}
\end{equation}
\begin{equation}
	M3\Pi(x_1, x_2) = \frac{(x_1 \cdot x_2)^{1/2}}{(x_1 \cdot x_2)^{1/2} + ((1 - x_1) \cdot (1 - x_2))^{1/2}} = \frac{\sqrt{x_1 \cdot x_2}}{\sqrt{x_1 \cdot x_2 + (1 - x_1) \cdot (1 - x_2)}}
\end{equation}

\subsection{Results}

We ran the program with both the Maximum and M3$\Pi$ aggregation operators and compared the results.

\begin{table}
	\centering
	\resizebox{\textwidth}{!}{%
		\begin{tabular}{|l|l|l|l|l|l|l|l|l|}
			\hline
			Bandwidth & Residual       & Data size & VM available & Number of        & Memory & Load & Processing & Fuzzy output                                     \\
			          & battery charge &           &              & concurrent users &        &      &            &                                                  \\
			\hline
			52        & 94             & 295       & 2            & 94               & 90     & 60   & 67.542     & 0.000/local_processing + 0.500/remote_processing \\
			99        & 30             & 65        & 8            & 7                & 4      & 56   & 67.228     & 0.000/local_processing + 1.000/remote_processing \\
			19        & 26             & 137       & 2            & 97               & 19     & 63   & 25.872     & 0.950/local_processing + 0.050/remote_processing \\
			13        & 49             & 393       & 13           & 77               & 98     & 88   & 42.664     & 0.650/local_processing + 0.350/remote_processing \\
			96        & 75             & 117       & 44           & 87               & 18     & 18   & 67.228     & 0.000/local_processing + 1.000/remote_processing \\
			34        & 22             & 193       & 38           & 64               & 10     & 65   & 44.635     & 0.600/local_processing + 0.400/remote_processing \\
			\hline
		\end{tabular}}
	\caption{Sample results from the fuzzy engine with the Maximum aggregation operator.}
	\label{tab:fuzzy-results}
\end{table}

\begin{table}
	\centering
	\resizebox{\textwidth}{!}{%
		\begin{tabular}{|l|l|l|l|l|l|l|l|l|}
			\hline
			Bandwidth & Residual       & Data size & VM available & Number of        & Memory & Load & Processing & Fuzzy output                                     \\
			          & battery charge &           &              & concurrent users &        &      &            &                                                  \\
			\hline
			52        & 94             & 295       & 2            & 94               & 90     & 60   & 67.542     & 0.000/local_processing + 0.630/remote_processing \\
			99        & 30             & 65        & 8            & 7                & 4      & 56   & 67.332     & 0.000/local_processing + 1.000/remote_processing \\
			19        & 26             & 137       & 2            & 97               & 19     & 63   & 26.878     & 0.950/local_processing + 0.063/remote_processing \\
			13        & 49             & 393       & 13           & 77               & 98     & 88   & 42.662     & 0.650/local_processing + 0.350/remote_processing \\
			96        & 75             & 117       & 44           & 87               & 18     & 18   & 67.332     & 0.000/local_processing + 1.000/remote_processing \\
			34        & 22             & 193       & 38           & 64               & 10     & 65   & 47.319     & 0.600/local_processing + 0.504/remote_processing \\
			\hline
		\end{tabular}}
	\caption{Sample results from the fuzzy engine with the 3$\Pi$ aggregation operator.}
	\label{tab:fuzzy-results-3pi}
\end{table}