\addcontentsline{toc}{chapter}{Abstract}

\hspace{0pt}
\vfill
\begin{otherlanguage}{french}
	\begin{abstract}
		\begin{otherlanguage}{french}
			La demande croissante d'applications et de services en temps réel a conduit à l'émergence de l'informatique
			FOG comme un paradigme prometteur pour étendre l'informatique en nuage jusqu'au bord du réseau.
			Cependant, la nature dynamique de l'environnement FOG pose plusieurs défis, tels que la planification
			des tâches, l'efficacité énergétique et la gestion des ressources. Dans ce contexte, de nombreux chercheurs
			ont proposé une large gamme d'algorithmes et de techniques pour optimiser les performances des systèmes en
			brouillard. Un travail en particulier servira de base à ce projet, nous utiliserons la logique floue pour
			construire un moteur de décision pour la délocalisation des tâches et améliorer ce moteur en utilisant des
			arbres de décision. Les résultats montrent que l'arbre de décision améliore les performances du moteur de
			logique floue en fournissant une décision de délocalisation plus précise. Malgré le fait de ne pas atteindre
			l'objectif initial, nous explorons des techniques potentielles pour créer un planificateur de tâches efficace
			pour les systèmes FOG.
		\end{otherlanguage}
	\end{abstract}
\end{otherlanguage}

\begin{abstract}
	With the increasing demand for real-time applications and services, fog computing has emerged as a promising paradigm
	to extend cloud computing to the edge of the network. However, the dynamic nature of the fog environment poses several
	challenges, such as task scheduling, energy efficiency, and resource management. In this context, many researchers have
	proposed a wide range of algorithms and techniques to optimize the performance of fog systems. One work in particular
	will the base of this project, we will use fuzzy logic to build a decision engine for task offloading and improve this
	engine using decision trees. Results show that the decision tree improves the performance of the fuzzy logic engine by
	providing a more accurate offloading decision. Despite not reaching the original objective, we explore potential
	techniques to create an efficient task scheduler for fog computing systems.
\end{abstract}
\vfill
\hspace{0pt}