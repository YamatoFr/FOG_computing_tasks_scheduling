\documentclass[12pt]{report}
\usepackage[utf8]{inputenc}
\usepackage[vmargin = 1.5cm, hmargin = 1.0cm]{geometry}
\usepackage[backend=biber,style=numeric,sorting=none]{biblatex}
\usepackage[dvipsnames]{xcolor}
\usepackage[T1]{fontenc}
\usepackage[english]{babel}
\usepackage[title]{appendix}
\usepackage[most]{tcolorbox}

\usepackage{fancyhdr, listings, minted, verbatim, indentfirst, underscore}
\usepackage{adjustbox, caption, graphicx, subcaption, threeparttable}
\usepackage{lastpage, longtable, tocloft, xcolor, bookmark}
\usepackage{amsmath, amssymb, amsfonts, mathtools}
\usepackage{mathptmx, lmodern}
\usepackage{hyperref, url}
\usepackage{titlesec}

% "csquotes should be loaded after fvextra to avoid a warning from the lineno package"
% but in this case the culprit is the "minted" package
\usepackage{csquotes}

%----------------------------------------------------------------------------------
%									  COMMANDES
%----------------------------------------------------------------------------------
\DeclarePairedDelimiter\abs{\lvert}{\rvert}%
\DeclarePairedDelimiter\norm{\lVert}{\rVert}%

\usemintedstyle{fruity}
\definecolor{bg}{HTML}{282828}

\setlength{\headheight}{15.35403pt}
\setlength{\parskip}{0.5em}

\setcounter{secnumdepth}{3}

\newlistof{links}{lks}{Liste des liens}
\newcommand\externallink[2]{%
\refstepcounter{links}%
\footnote{#1\url{#2}}%
\addcontentsline{lks}{links}{%
\protect\numberline{\thelinks}%
\protect{\url{#2}}}%
}

\hypersetup{
	colorlinks=true,
	linkcolor=blue,
	filecolor=magenta,      
	urlcolor=blue,
	citecolor=darkgray,
	pdftitle={Rapport de stage}
}

\newcommand\NB[1][0.3]{N\kern-#1em{B}}

%%Création commande pour insérer image avec nom d figure directement
%\newcommand{nomDeTaCommande}[nombreArguments]{CodeLaTeX}
%\insertImage[position]{image_path}{scale}{Titre_figure}{label}
\newcommand{\insertImage}[5][center]{
	\begin{#1}
		\includegraphics[scale=#3]{#2}
		\captionof{figure}{#4}
		\label{#5}
	\end{#1}
	}
	
	%%Création d'une nouvelle commande pour faire référence à une Figure
	%Exemple : \appelFigure{schema} donne : Figure 1 (en italique)
	\newcommand{\appelFigure}[1]{
		\textit{Figure \ref{#1}}
		}
		
		%%Création d'une nouvelle commande pour créer une barre horizontale
		\newcommand{\HRule}{\rule{\linewidth}{0.5mm}}
		
		\renewcommand{\contentsname}{Table of Contents}
		
		\DeclareMathOperator*{\argmin}{arg\,min}
		\DeclareMathOperator*{\argmax}{arg\,max}
		
		%----------------------------------------------------------------------------------
		%									INITIALISATIONS
		%----------------------------------------------------------------------------------
		\hbadness 11000
		\vbadness 11000
		\interlinepenalty 10000
		
		\title{{\huge \bfseries Fog computing task scheduling}\\[0.2cm]
		{\Large a multi-objective AI-based approach}}
		\author{Théo \textsc{FIGINI}\\M2 Computer Science\\Academic year 2023-2024}
		\date{June, 18\textsuperscript{th} 2024}
		\makeatletter
		\setcounter{chapter}{0}
		
		\pagestyle{fancyplain}
		\fancyhf{}
		\lhead{header}
		\rhead{\leftmark}
		\cfoot{Page \thepage /\pageref{LastPage}}
		\renewcommand{\headrulewidth}{0.5pt}
		\renewcommand{\footrulewidth}{0.5pt}
		\renewcommand{\plainheadrulewidth}{0.5pt}
		\renewcommand{\plainfootrulewidth}{0.5pt}
		\renewcommand\labelitemi{--}
		\renewcommand\topfraction{.9}
		\renewcommand\textfraction{0.35}
		\renewcommand\floatpagefraction{0.8}
		
		\bibliography{biblio}
		
		\counterwithout{figure}{chapter}
		
		
\begin{document}

\titleformat{\chapter}{\normalfont\large\bfseries}{\thechapter}{20pt}{}
\titleformat{\section}{\normalfont\large\bfseries}{\thesection}{1em}{}
\titleformat{\subsection}{\normalfont\normalsize\bfseries}{\thesubsection}{1em}{}

\titlespacing*{\chapter}{0pt}{-10pt}{25pt}

\begin{titlepage}
	\begin{center}
		\includegraphics[scale=1.80]{../images/logo_ufr_sen.png} \\[2cm]
		\hspace{2cm}

		\HRule \\[0.4cm]
		\@title \\[0.4cm]
		\HRule \\[1cm]

		\@author \\ [1.5cm]

		{\large Organisme d'accueil~: \textsl{Université des Antilles}} \\[1.5cm]

		\begin{minipage}{0.7\textwidth}
			\begin{center}
				Enseignant~: \\
				\hspace{0.2cm}  XX \textsc{XX}
			\end{center}
		\end{minipage}\\[3cm]

		\@date
	\end{center}
\end{titlepage}


\renewcommand{\thefigure}{\arabic{figure}}
\setcounter{figure}{0}

\addcontentsline{toc}{chapter}{Abstract}

\hspace{0pt}
\vfill
\begin{otherlanguage}{french}
	\begin{abstract}
		\begin{otherlanguage}{french}
			La demande croissante d'applications et de services en temps réel a conduit à l'émergence de l'informatique
			FOG comme un paradigme prometteur pour étendre l'informatique en nuage jusqu'au bord du réseau.
			Cependant, la nature dynamique de l'environnement FOG pose plusieurs défis, tels que la planification
			des tâches, l'efficacité énergétique et la gestion des ressources. Dans ce contexte, de nombreux chercheurs
			ont proposé une large gamme d'algorithmes et de techniques pour optimiser les performances des systèmes en
			brouillard. Un article de recherche en particulier a servi de base à ce projet,\cite{Hari-et-al-2018} grâce
			à la librairie python \textit{pyfuzzylite} nous avons utilisé la logique floue pour construire un moteur de
			décision pour la délocalisation des tâches et avons tenté d'implémenter l'opérateur d'agrégation \textit{Mean 3 Pi},
			sans succès. Nous avons utilisé le logiciel \textit{Weka} pour construire un arbre de décision qui nous a permis
			d'affiner notre moteur flou. Les résultats montrent que l'arbre de décision améliore les performances du moteur
			de logique floue en fournissant une décision de délocalisation plus précise. Malgré le fait de ne pas avoir pu
			atteindre l'objectif initial, nous explorons des techniques potentielles pour créer un planificateur de tâches
			efficace pour les systèmes FOG.
		\end{otherlanguage}
	\end{abstract}
\end{otherlanguage}

\begin{abstract}
	With the increasing demand for real-time applications and services, fog computing has emerged as a promising paradigm
	to extend cloud computing to the edge of the network. However, the dynamic nature of the fog environment poses several
	challenges, such as task scheduling, energy efficiency, and resource management. In this context, many researchers have
	proposed a wide range of algorithms and techniques to optimize the performance of fog systems. One paper in particular
	was the base of this project,\cite{Hari-et-al-2018} we used the python library \textit{pyfuzzylite} to implement
	a fuzzy logic engine for task offloading and attempt to implement the \textit{Mean 3 Pi} aggregation operator, without
	success. We used the \textit{Weka} software to build a decision tree that allowed us to refine our fuzzy engine.
	The results show that the decision tree improves the performance of the fuzzy logic engine by providing a more accurate
	offloading decision. Despite not reaching the initial goal, we explore potential techniques to create an efficient task
	scheduler for fog systems.
\end{abstract}
\vfill
\hspace{0pt}
\chapter{Introduction}

As the number of devices grows exponentially, the demand for computing resources increases. The traditional cloud
computing model is not sufficient to handle the massive amount of data generated by these devices. Fog computing is an
alternative model that extends the computing resources to the edge of the network, closer to the devices. This model
reduces the latency and bandwidth usage, and improves the overall performance of the system. However, the increasing
demand for computing resources also raises concerns about the energy consumption and the environmental impact of fog
computing systems.

In this context, task scheduling plays a crucial role in optimizing the performance of fog computing systems. Task
scheduling algorithms aim to allocate the available resources efficiently to the tasks, while still meeting the Quality
of Service (QoS) requirements of the users. Several task scheduling algorithms have been proposed in the literature,
ranging from simple heuristics to complex optimization techniques. However, most of these algorithms focus on
minimizing the energy consumption or maximizing the performance of the system, without considering the environmental
impact of the energy sources used to power the system.

In this paper, we will propose an AI-based task scheduling algorithm that integrate the use of green energy sources
while satisfying the QoS requirements of the users. The proposed algorithm will consider the availability of green
energy sources, mainly solar panels, and will schedule the tasks to maximize the use of green energy while minimizing
the energy consumption from the grid.

The rest of this paper is organized as follows. In Chapter 2, we present an overview of the existing research on task
scheduling, energy efficiency and AI in the context of fog computing. In Chapter 3, we describe the proposed task
scheduling algorithm and the integration of green energy sources. In Chapter 4, we present the experimental results
and the performance evaluation of the algorithm. Finally, in Chapter 5, we conclude the paper and discuss the future
work.
\chapter*{Related Work}
\label{chap:relatedwork}

There are many existing studies and researches on the topics task scheduling, energy efficiency and AI in the context
of fog computing. The concept of fog computing was introduced by Cisco in 2012 \cite{bonomi_et_al_2012}, and since
then, many researches have been conducted to improve the performance of fog computing systems and reduce their
environmental impact.

\section*{Task Scheduling in Fog Computing}

The task scheduling problem in fog computing has been widely studied in the literature. Many researchers have proposed
different algorithms and techniques to optimize the task scheduling process in fog computing systems. For example,
\cite{canali_lancellotti_2019} proposed a task scheduling algorithm based on the genetic algorithm to optimize the energy.
\chapter{Offloading using Fuzzy logic}
\label{chap:fuzzy}

\section{Introduction}

Introduced in 1965 by Lotfi Zadeh\cite{zadeh-1965}, fuzzy logic is based on a "degree of truth" instead of a finite
value, usually 0 or 1, it aims to represent the vagueness of human language and thought. Fuzzy systems are the means
to implement fuzzy logic, they are two types of fuzzy systems: Mamdani and Sugeno, both are similar but differ in the
way the output is determined. The most common one Mamdani and it's the one we will be using in this project. The
Mamdani fuzzy system follows three steps:

\begin{itemize}
	\item The inputs are fuzzified into fuzzy membership functions,
	\item a set of rules are applied to the fuzzy inputs to determine the fuzzy output,
	\item the fuzzy output is defuzzified to get a crisp value.
\end{itemize}

\subsection{Fuzzification}

Fuzzification is the process of converting a crisp value into a fuzzy value, this is done by assigning a membership
function to the input value. The membership function is a curve that defines how much the input value belongs to a
certain fuzzy set. The most common membership functions are the triangular and trapezoidal functions, they are defined
by three or four parameters respectively. The triangular function is defined by the parameters $a$, $b$ and $c$ and
is given by:

\begin{equation}
	\mu(x) = \begin{cases}
		0                   & \text{if } x \leq a,        \\
		\frac{x - a}{b - a} & \text{if } a \leq x \leq b, \\
		\frac{c - x}{c - b} & \text{if } b \leq x \leq c, \\
		0                   & \text{if } x \geq c.
	\end{cases}
\end{equation}

The trapezoidal function is defined by the parameters $a$, $b$, $c$ and $d$ and is given by:

\begin{equation}
	\mu(x) = \begin{cases}
		0                   & \text{if } x \leq a,        \\
		\frac{x - a}{b - a} & \text{if } a \leq x \leq b, \\
		1                   & \text{if } b \leq x \leq c, \\
		\frac{d - x}{d - c} & \text{if } c \leq x \leq d, \\
		0                   & \text{if } x \geq d.
	\end{cases}
\end{equation}

\subsection{Rule evaluation}

The rule evaluation is the process of determining the fuzzy output based on the fuzzy inputs and a set of rules. The
rules are defined by two parts: the "if" or antecedent part and the "then" or consequent part. The antecedent part
deals with inputs, it can either be a single input or a combination of inputs, the combination can be done using the
logical operators "and" and "or". The consequent part deals with the output. In the context of this project, the rules
are usually of the form "if Bandwidth is low then processing is local". The rules are generally defined by the user
and are based on their knowledge of the system.

\subsection{Aggregation}

The aggregation is the process of combining the fuzzy outputs from the rules to get a single fuzzy output. This process
relies on T-conorms, and must satisfy the following properties:

\begin{itemize}
	\item Commutativity: $x * y = y * x$,
	\item Associativity: $x * (y * z) = (x * y) * z$,
	\item Monotony: $x \leq y \implies x * z \leq y * z$,
	\item Neutrality of 0: $x * 0 = x$ for $x \in [0, 1]$.
\end{itemize}

They are also \textit{positive reinforcement} operators:

\begin{equation}
	f(x_1, \cdots, x_n) \leq max[x_i] \forall x_i \geq 0.5
\end{equation}

As opposed to T-norms which are \textit{negative reinforcement} operators:

\begin{equation}
	f(x_1, \cdots, x_n) \geq min[x_i] \forall x_i \leq 0.5
\end{equation}

The most common T-conorms are the maximum, the probabilistic sum and the bounded sum. They are defined as follows:

\begin{minipage}{0.9685\textwidth}
	\begin{equation}
		\text{Maximum: } x \oplus y = \max(x, y)
	\end{equation}
	\begin{equation}
		\text{Probabilistic sum: } x \oplus y = x + y - x \cdot y
	\end{equation}
	\begin{equation}
		\text{Bounded sum: } x \oplus y = \min(x + y, 1)
	\end{equation}
\end{minipage}

\subsection{Defuzzification}

Defuzzification is the process of converting a fuzzy output into a crisp value, there are several methods to do this,
the most common one is the centroid method. The centroid method calculates the center of mass of the fuzzy output, this
is done by taking the weighted average of the output values. The weighted average is calculated by taking the sum of
the product of the output value and its membership value divided by the sum of the membership values. The formula for
the centroid method is given by:

\begin{equation}
	y = \frac{\sum_{i} \mu_i \cdot y_i}{\sum_{i} \mu_i}
\end{equation}

Where $y$ is the crisp output, $\mu_i$ is the membership value of the output value $y_i$. The centroid method is the
most common method because it is simple and easy to implement. However, it is not always the best method, other methods
like the mean of maximum and the largest of maximum can be used depending on the application.

\section{Fuzzy system for offloading}

\subsection{Setting up the engine}

To demonstrate the use of fuzzy logic in offloading, we wrote two simple programs in Python. The goal was to recreate
the experiment done by Hari et al.\cite{Hari-et-al-2018}. The first program uses the pyfuzzylite\cite{fuzzylite} library
to implement a fuzzy engine with all the variables and rules needed to determine the offloading decision. The second
program uses the NumPy library to generate random values for the inputs and then uses the fuzzy engine to determine
the offloading decision. The fuzzy engine is defined by the variables shown in table \ref{tab:fuzzy-input} and
\ref{tab:fuzzy-output}.

\begin{table}[H]
	\centering
	\begin{tabular}{|c|c|c|c|c|}
		\hline
		Name                & Range    & Fuzzy set & Membership function & Parameters         \\
		\hline
		Bandwidth           & [0, 100] & bw_low    & trapezoidal         & 0, 20, 30, 40      \\
		(in Mbps)           &          & bw_med    & trapezoidal         & 35, 45, 60, 70     \\
		                    &          & bw_high   & trapezoidal         & 65, 75, 90, 100    \\
		\hline
		Data size           & [0, 600] & data_low  & trapezoidal         & 0, 0, 230, 360     \\
		(in KB)             &          & data_med  & trapezoidal         & 250, 350, 470, 590 \\
		                    &          & data_high & trapezoidal         & 450, 540, 600, 600 \\
		\hline
		Residual battery    & [0, 100] & bat_low   & trapezoidal         & 0, 0, 25, 35       \\
		charge (in \%)      &          & bat_med   & trapezoidal         & 25, 40, 60, 75     \\
		                    &          & bat_high  & trapezoidal         & 60, 75, 100, 100   \\
		\hline
		Load                & [0, 100] & load_low  & trapezoidal         & 0, 0, 25, 40       \\
		(in \%)             &          & load_med  & trapezoidal         & 35, 45, 60, 70     \\
		                    &          & load_high & trapezoidal         & 65, 80, 100, 100   \\
		\hline
		Memory              & [0, 100] & mem_low   & trapezoidal         & 0, 0, 25, 40       \\
		(in \%)             &          & mem_med   & trapezoidal         & 35, 45, 60, 70     \\
		                    &          & mem_high  & trapezoidal         & 65, 80, 100, 100   \\
		\hline
		Virtual machines    & [0, 50]  & vm_low    & trapezoidal         & 0, 0, 15, 20       \\
		available           &          & vm_med    & trapezoidal         & 15, 22, 37, 40     \\
		                    &          & vm_high   & trapezoidal         & 30, 35, 50, 50     \\
		\hline
		Number              & [0, 100] & user_low  & trapezoidal         & 0, 0, 25, 40       \\
		of concurrent users &          & user_med  & trapezoidal         & 30, 40, 60, 70     \\
		                    &          & user_high & trapezoidal         & 60, 75, 100, 100   \\
		\hline
	\end{tabular}
	\caption{Input variables for the fuzzy engine.}
	\label{tab:fuzzy-input}
\end{table}

\begin{table}[H]
	\centering
	\begin{tabular}{|c|c|c|c|c|}
		\hline
		Name                & Range    & Fuzzy set & Membership function & Parameters      \\
		\hline
		Offloading decision & [0, 100] & local     & trapezoidal         & 0, 12, 24, 48   \\
		                    &          & remote    & trapezoidal         & 36, 60, 72, 100 \\
		\hline
	\end{tabular}
	\caption{Output variable for the fuzzy engine.}
	\label{tab:fuzzy-output}
\end{table}

The original paper by Hari et al. did not provide all the rules used, so we had to come up with our own rules. We
established them based on our understanding of the system and the variables. Unlike the original paper, where the
rules only combined the bandwidth with one other variable, we decided to combine the bandwidth with all the relevant
variables. The rules are shown in table \ref{tab:fuzzy-rules}.

\begin{table}[H]
	\centering
	\resizebox{\textwidth}{!}{%
		\begin{tabular}{|l|l|}
			\hline
			   & Rules                                                                                                  \\
			\hline
			R1 & IF Bandwidth is bw_low                                                                                 \\
			   & THEN Processing local_processing                                                                       \\
			\hline
			R2 & IF Bandwidth is not bw_low and Datasize is not data_low                                                \\
			   & THEN Processing is remote_processing                                                                   \\
			\hline
			R3 & IF Bandwidth is not bw_low and Datasize is data_low and (Load is load_high or Memory is mem_low)       \\
			   & THEN Processing is remote_processing                                                                   \\
			\hline
			R4 & IF Bandwidth is not bw_low and Datasize is data_low and (NB_concurrent_users is user_low or Memory is
			mem_low)                                                                                                    \\
			   & THEN Processing is remote_processing                                                                   \\
			\hline
			R5 & IF Bandwidth is not bw_low and Datasize is data_low and NB_concurrent_users is not user_low and Memory
			is not mem_low and Load is load_low                                                                         \\
			   & THEN Processing is local_processing                                                                    \\
			\hline
			R6 & IF Bandwidth is not bw_low and Datasize is data_low and NB_concurrent_users is not user_low and Memory
			is not mem_low and Load is not load_low                                                                     \\
			   & THEN Processing is remote_processing                                                                   \\
			\hline
		\end{tabular}}
	\caption{Rules for the fuzzy engine.}
	\label{tab:fuzzy-rules}
\end{table}

From these rules we can see that two variables have no impact on the offloading decision, the residual battery charge
and the number of virtual machines available. So we decided to remove them from the fuzzy engine.

\subsection{Mean 3\texorpdfstring{$\Pi$}{Pi} aggregation operator (M3\texorpdfstring{$\Pi$}{Pi})}

The standard 3$\Pi$ operator was introduced by Yager and Rybalov \cite{yager-rybalov-1998} in 1988, it is a
generalization of the probabilistic sum. The 3$\Pi$ operator is defined by the following formula:

\begin{equation}
	\Pi(x_1, \cdots, x_n) = \frac{\Pi_{j=1}^n x_j}{\Pi_{j=1}^n x_j + \Pi_{j=1}^n (1 - x_j)}
\end{equation}

Unlike T-norms and T-conorms, the 3$\Pi$ operator is \textit{full reinforced} which means that it satisfies the
properties of both positive
and negative reinforcement operators.

We decided to implement a M3$\Pi$ aggregation operator which is derived from the 3$\Pi$ operator, it is defined by the
following formula:

\begin{equation}
	M3\Pi(x_1, \cdots, x_n) = \frac{\Pi_{j=1}^n (x_j)^{(1/n)}}{\Pi_{j=1}^n (x_j)^{(1/n)} + \Pi_{j=1}^n (1 - x_j)^{(1/n)}}
\end{equation}

The goal was to improve the decision-making process. However, we were unable to implement the M3$\Pi$ operator as it
would require to modify a large part of the \textit{pyfuzzylite} library. We decided to shift our focus towards other
parts of the project.

\subsection{Partial results}

Despite the lack of the M3$\Pi$ operator, we still ran the fuzzy engine with three different aggregation operators:
maximum, probabilistic sum and bounded sum. Samples of the results are shown in tables \ref{tab:fuzzy-results-max},
\ref{tab:fuzzy-results-prob} and \ref{tab:fuzzy-results-bounded}.

\begin{table}[H]
	\centering
	\resizebox{\textwidth}{!}{%
		\begin{tabular}{|l|l|l|l|l|l|l|l|l|}
			\hline
			Bandwidth & Data size & Number of        & Memory & Load & Processing & Fuzzy output                                     \\
			          &           & concurrent users &        &      &            &                                                  \\
			\hline
			30        & 180       & 42               & 38     & 66   & 21.575     & 1.000/local_processing + 0.000/remote_processing \\
			77        & 462       & 44               & 96     & 8    & 67.539     & 0.000/local_processing + 1.000/remote_processing \\
			3         & 21        & 45               & 75     & 50   & 60.732     & 0.150/local_processing + 0.850/remote_processing \\
			38        & 229       & 51               & 31     & 66   & 57.565     & 0.200/local_processing + 0.600/remote_processing \\
			50        & 302       & 17               & 96     & 92   & 68.116     & 0.000/local_processing + 0.554/remote_processing \\
			\hline
		\end{tabular}}
	\caption{Sample results from the fuzzy engine with the Maximum aggregation operator.}
	\label{tab:fuzzy-results-max}

	\bigskip

	\centering
	\resizebox{\textwidth}{!}{%
		\begin{tabular}{|l|l|l|l|l|l|l|l|l|}
			\hline
			Bandwidth & Data size & Number of        & Memory & Load & Processing & Fuzzy output                                     \\
			          &           & concurrent users &        &      &            &                                                  \\
			\hline
			30        & 180       & 42               & 38     & 66   & 21.575     & 1.000/local_processing + 0.000/remote_processing \\
			77        & 462       & 44               & 96     & 8    & 67.539     & 0.000/local_processing + 1.000/remote_processing \\
			3         & 21        & 45               & 75     & 50   & 60.443     & 0.150/local_processing + 0.850/remote_processing \\
			38        & 229       & 51               & 31     & 66   & 61.27      & 0.200/local_processing + 0.904/remote_processing \\
			50        & 302       & 17               & 96     & 92   & 68.314     & 0.000/local_processing + 0.863/remote_processing \\
			\hline
		\end{tabular}}
	\caption{Sample results from the fuzzy engine with the Probabilistic sum
		aggregation operator (note: the library calls it Algebraic sum).}
	\label{tab:fuzzy-results-prob}

	\bigskip

	\centering
	\resizebox{\textwidth}{!}{%
		\begin{tabular}{|l|l|l|l|l|l|l|l|l|}
			\hline
			Bandwidth & Data size & Number of        & Memory & Load & Processing & Fuzzy output                                     \\
			          &           & concurrent users &        &      &            &                                                  \\
			\hline
			30        & 180       & 42               & 38     & 66   & 21.575     & 1.000/local_processing + 0.000/remote_processing \\
			77        & 462       & 44               & 96     & 8    & 67.539     & 0.000/local_processing + 1.000/remote_processing \\
			3         & 21        & 45               & 75     & 50   & 60.275     & 0.150/local_processing + 0.850/remote_processing \\
			38        & 229       & 51               & 31     & 66   & 61.855     & 0.200/local_processing + 1.000/remote_processing \\
			50        & 302       & 17               & 96     & 92   & 67.903     & 0.000/local_processing + 1.000/remote_processing \\
			\hline
		\end{tabular}}
	\caption{Sample results from the fuzzy engine with the Bounded sum aggregation operator.}
	\label{tab:fuzzy-results-bounded}
\end{table}

\section{AI-based task scheduler}
\chapter{Conclusion}
\label{chap:conclusion}

Throughout this project, we have shown the part that FOG computing occupies in our daily life and the importance
of scheduling in such environments. We created a fuzzy logic offloading system that can be used to offload tasks
from a mobile device to a FOG node or the cloud. We then improved upon this system by using decision trees to
refine the decision-making process. With the help of the decision tree, we saw improvements in the offloading
decision taken by the fuzzy system.

While we were unable to reach our original objective of creating a task scheduler, we nonetheless have room for
improvement. Using machine learning techniques and other optimization algorithms could help us build a more
efficient scheduler. We could also improve the system by adding more features, such as integrating the fuzzy
system into the scheduler itself. This would allow the system to make decisions based on the current state of the
network and the device, rather than relying on a pre-defined set of rules.

It was also a great learning experience, as we were able to explore new and existing technologies, and apply them
to a real-world problem. Overall, we believe that our system is a good starting point for future research in this
area. We hope that our work will inspire others to explore the possibilities of FOG computing and help to create
a more efficient and reliable offloading system for mobile devices.

If we were to continue this project, we would focus on improving the performance of the system by leveraging
machine learning techniques and other optimization algorithms. We would also explore the possibility of integrating
the fuzzy system into the scheduler itself, to make the system more adaptive and responsive to changes in the
network and device state. Finally, we would conduct more experiments to evaluate the performance of the system
and compare it with other offloading systems to validate our approach.

\printbibliography
\end{document}