\chapter{Conclusion}
\label{chap:conclusion}

Throughout this project, we have shown the part that FOG computing occupies in our daily life and the importance
of scheduling in such environments. We created a fuzzy logic offloading system that can be used to offload tasks
from a mobile device to a FOG node or the cloud. We then improved upon this system by using decision trees to
refine the decision-making process. With the help of the decision tree, we saw improvements in the offloading
decision taken by the fuzzy system.

While we were unable to reach our original objective of creating a task scheduler, we nonetheless have room for
improvement. Using machine learning techniques and other optimization algorithms could help us build a more
efficient scheduler. We could also improve the system by adding more features, such as integrating the fuzzy
system into the scheduler itself. This would allow the system to make decisions based on the current state of the
network and the device, rather than relying on a pre-defined set of rules.

It was also a great learning experience, as we were able to explore new and existing technologies, and apply them
to a real-world problem. Overall, we believe that our system is a good starting point for future research in this
area. We hope that our work will inspire others to explore the possibilities of FOG computing and help to create
a more efficient and reliable offloading system for mobile devices.

If we were to continue this project, we would focus on improving the performance of the system by leveraging
machine learning techniques and other optimization algorithms. We would also explore the possibility of integrating
the fuzzy system into the scheduler itself, to make the system more adaptive and responsive to changes in the
network and device state. Finally, we would conduct more experiments to evaluate the performance of the system
and compare it with other offloading systems to validate our approach.